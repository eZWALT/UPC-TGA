\documentclass[12pt]{article}
\usepackage{graphicx}
\usepackage{amsmath}
\usepackage{placeins}
\usepackage{float}
\usepackage[bottom]{footmisc}
\usepackage{subcaption}
\usepackage{booktabs}  % Per a l'ús de línies horitzontals millorades (\toprule, \midrule, \bottomrule)
\usepackage{hyperref}
\usepackage{algorithm}
\usepackage{algpseudocode}
\usepackage{geometry}
\usepackage{caption}
\usepackage{multirow}



\begin{document}

\begin{titlepage}
	\centering
	\includegraphics[width=0.3\textwidth]{university.png}\par\vspace{1cm}
	{\scshape\LARGE Targetes Gràfiques i Acceleradors \par}
	\vspace{1cm}
	{\scshape\Large Examen Teoria \par}
	\vspace{1.5cm}
	{\Large\itshape Walter J.Troiani\par}
	\vfill
	Prof: Agustín \& Dani \par
    \vspace{1cm}
    2/12/2023 - 2023/24 Q1\par
	\vspace{1cm}

\end{titlepage}

\newpage

\section{Pregunta 1}

El pipeline gràfic es en paraules simples, la funció o procés sequencial utilitzat per a "rendertizar" (pintar) informació gràfica en 2/3 dimensions en dispositius de visualització (pantalles)
bidimiensionals. Aquest algoritme amb etapes ben diferenciades ha sigut essencial en la revolució del segle passat dels gràfics computacionals, que poc a poc han anat evolucionant tenint un enorme impacte
en l'evolució dels computadors, per la revolució economica que ha causat en l'industria dels videojocs, sanitat i en l'animació cinematogràfica. Aquest mateix algorisme ha estat implementat
en diverses API/Plataformes gràfiques com OpenGL de Grup Khronos, DirectX de Microsoft o bé Vulkan i en totes implementacions es busca explotar el paralelisme, oferir una plataforma unificada
amb la qual els usuaris puguin crear comunitat i aixi aquesta millori amb el temps, ocultar la latencia dels moviments de memoria i oferir resultats ràpids explotant el hardware especialitzat.
Aquestes implementacions parteixen de l'idea que algunes etapes són indispensables i per tant fixes i altres es deixen programables a disposició de l'usuari de manera intuitiva, per a poder
adaptar la renderizació a l'aplicació/simulació o cas d'ús desitjat.
\\

La primera gran etapa (fixa) es tracta del VERTEX PULLER, on es fa les transferencies a memoria necessàries dels models i demés informació en sentit CPU -> GPU, aquí es llegeix tota la informació gràfica 
bàsica de cada vertex que composa cada model, com la posició (vector 3d XYZ), el color (vector 4d RGBA), les coordenades de textura (vector 2d ST) i vectors normals (3d XYZ) entre altres. A continuació 
ve una etapa programable importantíssima, el VERTEX PROCESSING/SHADER on l'usuari pot especificar quina transformació vectorial es fará a cada vertex (Normalment es fan els càlculs de totes les transformacions
com translacions/rotacions/escalats, cambiantlos a un altre espai de coordenades i també els càlculs de la posició de la càmera i iluminació incident sobre aquests vertexs). Aquesta etapa també sol enviar 
variables de sortida , per a cada vertex com la posició i colors finals, vectors normals transformats o les coordenades de textura, per a que la seguent etapa es pugui emprar.
\\

Seguidament passen a l'etapa de teselació, una funcionalitat que es va afegir fa pocs anys en OpenGL/DirectX per a multiplicar el nombre de vertexs dels models i així millorar la qualitat d'aquests. 
Consta de 3 sub-etapes com Hull Shader, Tessellator i Domain Shader, els quals els shaders també són programables, però com aquests són molt recents no els considerarem com a part del pipeline tradicional.
\\

A continuació s'inicia el processament de primitives, primerament amb una funció fixa de GENERACIÓ DE PRIMITIVES, a partir de la llista de vertexs o índexos dels vertexs i la primitiva desitjada
(Per primitiva s'entén figura bàsica com linia, triangle, cuadrat...) genera una llista de primitives i una associació entre quins vertexs van a cada primitiva. Després aquestes primitives serán 
enviades a la etapa de funció fixa també de PROCESSAMENT DE PRIMITIVES previes a la rasterizació, com el CLIPPING (eliminació de primitives que no apareixin en el area de visió) o bé CULLING
(Fer ús d'un buffer en la memòria de la tarjeta gràfica que guarda les components Z de cada vertex, per a poder descartar els vertexs que quedaran tapats per altres davant seu) d'entre altres possibles.
També fa poc es va afegir una etapa programable de GEOMETRY SHADERS per que l'usuari pugui manipular també les primitives
\\

Un cop acabat el processament de primitives es passa per una etapa de funció fixa crucial, el algorisme de RASTERITZACIÓ, que converteix aquestes primitives a una matriu de píxels
(Resolució de pantalla), aleshores el procediment decideix mitjançant la cámara i els objectes quin es el color de cada píxel i el guarda en el FRAMEBUFFER. Seguidament pasen el FRAGMENT SHADER,
 una etapa programable , pero opcional (Ja que amb el Raster es podria considerar suficient), que donat un pixel i la seva posició i possiblement altres variables com les coordenades de textura...
  Determina el color, la possible iluminació per fragment, el ús de textures i possibles manipulacions dels atributs profunditat i stencil (Per més profunditat sobre l'us de stencil consultar la guía de khronos).
Aquesta etapa és costosa degut al enorme nombre de fragments que hi ha en una pantalla (Full HD es 1920x1080, del ordre $10^6$), pero com cap fragment depen del altre es completament paralelitzable.
\\

Un cop finalizat les etapes més importants, amb el FRAMEBUFFER ja llest per mostrar a pantalla, es fa un seguit d'operacions opcionals anomenades PIXEL OPERATIONS, que consten en filtres
 o combinacions de píxels, per a aconseguir de manera eficient efectes especials si son desitjats i s'executen de manera paralela en les unitats ROP (Render Output Unit). Algunes de les operacions més importants són:
Scissor test (Es descarten fragments en funció de l'area de renderització), Alpha Test (Es fa servir la component alfa de RGBA per descartar fragments transparents), Stencil Test (Emprant el stencil buffer, serveix per
aplicar efectes especials com siluetes als models o ombres), Z test (Per descartar fragments que queden ocults per profunidatat, emprant el Z-Buffer com a base) o Alpha Blending ().
\\

Un cop acabades les operacions es mostra el frame buffer per pantalla , resultant en l'imatge renderitzada en 2D amb tots els efectes i shaders aplicats.
\\ 

Aquest procediment ha estat estudiat durant decàdes i ha sigut objecte de nombroses optimitzacions i millores per part de la enorme comunitat i empreses que han estat darrere per inventar els gràfics 
del demà. L'industria del cinema ha prosperat gràcies als avenços en aquest camp, empreses com Disney, Pixar, Dreamworks i les targetes gràfiques van apareixer i han seguit evolucionant de manera ultra-especifica 
 per optimitzar aquest procediment de manera minuciosa (Unitats especialitzades pel costos ray-tracing, unitats també uniques per les Pixel Operations, Amples de banda creixents per poder apagar el coll d'ampolla que suposen 
les transferencies de memòria, un nombre enorme de unitats de shading unificades per poder explotar el potencial paralelisme ...) i són una autèntica meravella electrònica, fruit de les millors ments, anys de esforços gegantins i
competitivat elevada i ganes de portar la tècnologia més enllà.

\section{Pregunta 2}

\section{Pregunta 3}
El mercat de les tarjetes gràfiques sempre ha sigut un entorn molt competitiu on unes poques empreses han conseguit emportar-se tot el pastís (Degut en gran part a la seva supremacía d'inovacció) i les altres que han intentat entrar
 o bé han fracasat o bé han sigut comprades i assimilades per aquests gegants tècnologics que són NVIDIA, AMD o INTEL. La distribució dels guanys i del mercat segueix una mena de distribució zipfiana, es a dir, els primers s'ho emporten 
gairebé tot, els segons molt, els tercers un bon tros, però de manera exponencial van reduint-se els recursos. NVIDIA és el indisputable fabricant més important a dia d'avui de tarjetes gràfiques, tant pels avenços en el món dels gràfics 
com en els camps de la computació científica, inteligencia artificial, simulació... Ha estat pioner en la recerca i optimització d'aquets dispositius, la pipeline gràfica i la supercomputació, tal i com es pot veure en les 
unitats especialitzades de Ray-Tracing, els Tensor-cores especialitzats en la computació d'operacions matricials petites en una única operació... Cosa que l'ha diferenciat i posat per sobre en la carrera amb els seus altres 2 
principals competidors. A més a més les optimitzacions específiques i influencia que han tingut en el món dels videojocs ha estat probablement un dels factors económics més determinants en la seva popularitat i supremacía. 
\\

Per demostrar que realment Nvidia és el fabricant més rellevant cal també observar les dades, si mirem les mides de les empreses i la seva facturació... 
\\ 

Pel que fa a la tarjeta gràfica més important de l'historia no és fàcil d'identificar, ja que hi han hagut diversos dispositius que han marcat un abans i un després en aquest camp. 3Dfx Voodoo va estar una de les més remarcables gràfiques
(A pesar de que es considerava un accelerador) degut a ser la primera en implementar gràfics en 3D en el món dels jocs. També es important mencionar Hercules, una de les primeres gràfiques dels 80's que permitia visualitzar jocs senzills del estil pixel art,
a pesar de les series limitacions que patia. Pero realment la paraula "tarjeta gràfica" va ser acunyada gràcies a una dels primers models de Nvidia, la GeForce 256. Aquesta gràfica llançada en 1999 comptaba 4 fragment shaders, Texture units i ROP's únicament i 
compatava un amb ample de banda mai vist anteriorment de 4.8 GB/segon gràcies a la nova memòria gràfica DDR, que permitia executar les primeres versions de DirectX(7.0) i OpenGL(1.2).
\\

També es podrien destacar varis models de la saga TITAN de nvidia, una saga que va començar a partir del 2015 i que va donar varis models increíblement rellevants degut al salt de prestacions que suposaven i de les novetats tècnologiques
que introduien al món de les gràfiques. Titan V va introduir els famosos Tensor Cores, el format de memòria HBMS
\\

Pero sense dubte algun, la 2080....




\section{Pregunta 4}

\section{Pregunta 5}


\section{Pregunta Doble: Tensor Cores}

\section{Questionari}

\subsection{a}

\subsection{b}

\subsection{c}

\subsection{d}

\subsection{e}

\subsection{f}

\subsection{g}

\subsection{h}

\subsection{i}

\subsection{j}









\end{document}

